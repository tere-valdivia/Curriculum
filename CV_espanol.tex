%%%%%%%%%%%%%%%%%%%%%%%%%%%%%%%%%%%%%%%%%
% "ModernCV" CV and Cover Letter
% LaTeX Template
% Version 1.1 (9/12/12)
%
% This template has been downloaded from:
% http://www.LaTeXTemplates.com
%
% Original author:
% Xavier Danaux (xdanaux@gmail.com)
%
% License:
% CC BY-NC-SA 3.0 (http://creativecommons.org/licenses/by-nc-sa/3.0/)
%
% Important note:
% This template requires the moderncv.cls and .sty files to be in the same 
% directory as this .tex file. These files provide the resume style and themes 
% used for structuring the document.
%
%%%%%%%%%%%%%%%%%%%%%%%%%%%%%%%%%%%%%%%%%

%----------------------------------------------------------------------------------------
%   PACKAGES AND OTHER DOCUMENT CONFIGURATIONS
%----------------------------------------------------------------------------------------

\documentclass[11pt,a4paper,verdana]{moderncv} % Font sizes: 10, 11, or 12; paper sizes: a4paper, letterpaper, a5paper, legalpaper, executivepaper or landscape; font families: sans or roman



\moderncvstyle{classic} % CV theme - options include: 'casual' (default), 'classic', 'oldstyle' and 'banking'
\moderncvcolor{orange} % CV color - options include: 'blue' (default), 'orange', 'green', 'red', 'purple', 'grey' and 'black'



\usepackage{lipsum} % Used for inserting dummy 'Lorem ipsum' text into the template

\usepackage[scale=0.8]{geometry} % Reduce document margins
%\setlength{\hintscolumnwidth}{3cm} % Uncomment to change the width of the dates column
%\setlength{\makecvtitlenamewidth}{10cm} % For the 'classic' style, uncomment to adjust the width of the space allocated to your name

%----------------------------------------------------------------------------------------
%   NAME AND CONTACT INFORMATION SECTION
%----------------------------------------------------------------------------------------

\firstname{Mar\'ia Teresa} % Your first name
\familyname{Valdivia Mena} % Your last name

% All information in this block is optional, comment out any lines you don't need

\newcommand{\cvdoublecolumn}[2]{%
  \cvitem[0.75em]{}{%
    \begin{minipage}[t]{\listdoubleitemmaincolumnwidth}#1\end{minipage}%
    \hfill%
    \begin{minipage}[t]{\listdoubleitemmaincolumnwidth}#2\end{minipage}%
    }%
}


\newcommand{\cvreference}[7]{%
    \textbf{#1}\newline% Name
    \ifthenelse{\equal{#2}{}}{}{\addresssymbol~#2\newline}%
    \ifthenelse{\equal{#3}{}}{}{#3\newline}%
    \ifthenelse{\equal{#4}{}}{}{#4\newline}%
    \ifthenelse{\equal{#5}{}}{}{#5\newline}%
    \ifthenelse{\equal{#6}{}}{}{\emailsymbol~\texttt{#6}\newline}%
    \ifthenelse{\equal{#7}{}}{}{\phonesymbol~#7}}


\title{Curriculum Vitae}
\address{Giessenbachstrasse 1, 85748 Garching, Alemania}
\mobile{+49 1520 5607211}
\email{mvaldivi@mpe.mpg.de}
\homepage{https://tere-valdivia.github.io/}% The first argument is the url for the clickable link, the second argument is the url displayed in the template - this allows special characters to be displayed such as the tilde in this example
\extrainfo{}
%\photo[70pt][0.4pt]{picture} % The first bracket is the picture height, the second is the thickness of the frame around the picture (0pt for no frame)
%\quote{"A witty and playful quotation" - John Smith}

%----------------------------------------------------------------------------------------

\begin{document}

\makecvtitle % Print the CV title

%----------------------------------------------------------------------------------------
%   EDUCATION SECTION
%----------------------------------------------------------------------------------------
\section{Datos Personales}
\cvitem{RUN}{19076878-3}
\cvitem{Fecha de nacimiento}{28 de enero, 1995}
\cvitem{Nacionalidad}{Chile}
\cvitem{Residencia}{Munich, Alemania}

\section{Educaci\'on}
\cventry{2020 - }{Doctorado en Astrof\'isica}{Instituto Max Planck de F\'isica Extraterrestre - Ludwig Maximilians Universit\"{a}t}{Alemania}{A traves del International Max Planck Research School}{}
\cventry{2018 - 2020}{Mag\'ister en Ciencias Menci\'on Astronom\'ia}{Universidad de Chile}{Chile}{}{}
\cventry{2016 - 2017}{Minor en Computaci\'on Cient\'ifica}{Universidad de Chile}{Chile}{}{}
\cventry{2014 - 2017}{Licenciatura en Ciencias Menci\'on Astronom\'ia}{Universidad de Chile}{Chile}{}{}
 % Arguments not required can be left empty
\cventry{2012 - 2013}{Bachillerato Internacional (IB)}{Colegio Santiago College}{Chile}{\break Cursos Nivel Superior: F\'isica, Matem\'aticas e Ingl\'es (Literatura)}{} %Cursos Nivel Medio: Lenguaje Espa\~nol, Biolog\'ia y Psicolog\'ia
\cventry{2000 - 2013}{Educaci\'on B\'asica y Media}{Colegio Santiago College}{Chile}{}{}


%----------------------------------------------------------------------------------------
%   WORK EXPERIENCE SECTION
%----------------------------------------------------------------------------------------
\section{Experiencia en investigaci\'on}

Actualmente trabajo en el Centro de Estudios de Astroqu\'imica del Instituto Max Planck de F\'isica Extraterrestre bajo la supervisi\'on de la Dra. Paola Caselli y el Dr. Jaime Pineda. Mi investigaci\'on trata sobre las propiedades fisico-qu\'imicas del material que envuelve objetos estelares j\'ovenes (young stellar objects o YSOs). En particular, mi trabajo consiste en encontrar formas diferentes a la ca\'ida sim\'etrica de material desde el n\'ucleo gaseoso de donde nacen las estrellas hacia sus discos protoestelares, para ver qu\'e efecto tiene el colapso asim\'etrico en la formaci\'on de estrellas de baja masa. Para mi trabajo, uso observaciones de diferentes l\'ineas moleculares obtenidas con telescopios tanto de una antena (por ejemplo el IRAM 30m) como radiotelescopios interferom\'etricos (como ALMA y NOEMA).
\\
Realic\'e mi Mag\'ister en el Departamento de Astronom\'ia de la Universidad de Chile bajo la supervisi\'on de la Dra. M\'onica Rubio. Mi investigaci\'on trat\'o sobre las propiedades f\'isicas de nubes moleculares en medio ambientes pobres en metales, en particular en las Nubes de Magallanes. Para esto, us\'e observaciones de los radiotelescopios Atacama Large Millimeter/Submillimeter Array (ALMA) y Atacama Pathfinder Explorer (APEX), complementados con datos de archivo de los telescopios espaciales Herschel y Spitzer.
\newpage

\subsection{Publicaciones}
\cventry{}{PRODIGE - Envelope to disk with NOEMA I. The first streamer feeding a Class I protostar}{Valdivia-Mena, M.~T. , Pineda, J.~E., Segura-Cox, D.~M., Caselli, P. et. al.}{presentado a Astronomy \& Astrophysics en febrero, 2022}{}{}
\cventry{Septiembre 2020}{ALMA resolves molecular clouds in metal-poor Magellanic Bridge A}{Valdivia-Mena, M.~T., Rubio, M., Bolatto, A.~D, Salda{\~n}o, H.~P. and Verdugo, C.}{Astronomy \& Astrophysics, 641, A97}{}{} 
%\cventry{}{Submillimeter Excess in Magellanic Bridge A}{Valdivia, M.~T. et. al. in prep.}{}{}{} %
\cventry{Agosto 2019}{Molecular Clouds resolved with ALMA $^{12}$CO J=1-0 and J=2-1 observations towards the Magellanic Bridge}{Valdivia, M.~T., Mu{\~n}oz, M., Rubio, M. and Salda{\~n}o, H.}{Bolet\'in de la Asociaci\'on Argentina de Astronom\'ia, 61, 134-136}{}{} %


\subsection{Participaci\'on en Congresos}
\cventry{2 al 4 de marzo, 2022}{Meeting of ALMA Young Astronomers}{Virtual}{European ALMA Regional Centre Network}{\textbf{Contribuci\'on Oral:}  \textit{Rivers in the sky: streamers discovered towards two Class I sources in Perseus}}{}
\cventry{4 al 5 de noviembre, 2021}{12th IMPRS Students Symposium}{Instituto Max Planck de Astrof\'isica}{Garching, Alemania}{\textbf{Miembro organizador y contribuci\'on oral:} \textit{River in the sky: A streamer feeding a Class I protostar}}{}
\cventry{4 al 29 de octubre, 2021}{Gaps, Rings, Spirals, and Vortices: Structure Formation in Planet-Forming Disks}{Munich Institute for Astro- and Particle Physics}{Garching, Alemania}{\textbf{Contribuci\'on Oral:} \textit{A Class I protostar with a high streamer mass infall rate}}{}
\cventry{28 de junio al 2 de julio, 2021}{European Astronomical Society Annual Meeting 2021}{Virtual}{Leiden, Pa\'ises Bajos}{\textbf{Contribuci\'on Oral:} \textit{River in the sky: A streamer feeding a Class I protostar}}{}
\cventry{3 al 9 de noviembre, 2019}{Latin American Regional IAU Meeting 2019}{Hotel Antofagasta}{Antofagasta, Chile}{\textbf{Contribuci\'on Oral:} \textit{ALMA observations of molecular tracers in R136 in 30Dor: How does cold gas survive?}}{}
\cventry{14 al 18 de octubre, 2019}{ALMA2019: Science Results and Cross-Facility Synergies}{T-Hotel Cagliari}{Cagliari, Italia}{\textbf{Poster (co-autora)}: \textit{Cold Gas Survives near R136 in 30 Doradus}}{}
\cventry{7 al 12 de octubre, 2018}{Second Binational AAA-SOCHIAS Meeting}{Hotel Club La Serena}{La Serena, Chile}{\textbf{Poster}: \textit{Molecular Clouds Resolved in $^{12}CO(1-0)$ and  $^{12}CO(2-1)$ in the Magellanic Bridge}}{}

\subsection{Experiencia en Observaci\'on Astron\'omica}

\textbf{IRAM 30m}: He sido observadora en dos ocasiones de forma remota: para el proyecto 124-21 (2022A) donde soy investigadora principal, y para el proyecto 090-21 (2021B, PI: Jaime Pineda), donde fui observadora asistente.

\textbf{APEX}: He sido observadora para tiempo chileno en el telescopio APEX en tres periodos: junio del 2017 (2017A), abril del 2018 (2018A) y mayo del 2019 (2019A). Soy investigadora principal del proyecto C104-9709 (2019B).


\section{Cursos adicionales}
\cventry{15 al 23 de noviembre, 2021}{10th IRAM 30-meter School on Millimeter Astronomy}{Workshop Virtual}{Institut de Radioastronomie Millim\'etrique}{}{}
\cventry{14 al 19 de marzo, 2021}{2021 Submillimeter Array Interferometry School}{Virtual Workshop}{Harvard \& Smithsonian Center for Astrophysics, in conjunction with the ASIAA and the University of Hawaii}{}{}
\cventry{7 al 11 de octubre, 2019}{European Radio Interferometry School 2019}{Quality Hotel Panorama, Onsala Space Observatory}{Gothenburg, Sweden}{}{}
\cventry{1 al 5 de octubre, 2018}{II International Astrochemistry School}{Universidad Aut\'onoma de Chile}{Santiago, Chile}{}{}
\cventry{20 al 29 de agosto, 2018}{La Serena School for Data Science: Applied Tools for Data Driven Sciences}{Association of Universities for Research in Astronomy (AURA) Campus}{La Serena, Chile}{}{}%\textbf{Proyectot}: \textit{Global Obesity Prevalence: Important Factors and Characteristics}
\cventry{6 al 7 de marzo, 2017}{Workshop de Estudiantes de Astronom\'ia}{Observatorio Astron\'omico Nacional}{Santiago, Chile}{}{}% \textbf{Contribuci\'on oral}: \textit{Characterizations of CO Emissions in MagBridgeC}


\section{Experiencia Laboral y Acad\'emica}
\cventry{\small{Septiembre 2018 - Julio 2019}}{Profesora Auxiliar del curso de Astronom\'ia Experimental}{Universidad de Chile}{}{}{\break El/ la auxiliar hace clases demostrativas con ejercicios complementarios a la materia de c\'atedra, se encarga de guiar el trabajo en los laboratorios de computaci\'on para el desarrollo de tres proyectos relacionados con astronom\'ia experimental y eval\'ua los informes resultantes de cada grupo.}
\cventry{\small{Marzo 2019 - Julio 2019}}{Profesora Auxiliar del curso de Introducci\'on a la Cosmolog\'ia}{Universidad de Chile}{}{}{\break El/ la auxiliar hace clases demostrativas con ejercicios complementarios a la materia de c\'atedra y se encarga de corregir las evaluaciones.}
\cventry{\small{Julio 2018 - Julio 2019}}{Profesora particular}{Santiago, Chile}{}{}{Apoyo para el Bachillerato Internacional (F\'isica HL)}
\cventry{\small{Junio 2018 - Noviembre 2018}}{\textbf{Asesora cient\'ifica}}{Liceo Polivalente San Jos\'e de Maipo}{Festival de Teatro y Ciencias}{Par Explora Sur Oriente, CONICYT}{\break El/la asesor/a cient\'ifico/a se encarga de guiar a los alumnos participantes de la competencia con respecto a los contenidos cient\'ificos que deseen abordar en su obra. }%Debe fomentar el pensamiento cr\'itico y la b\'usqueda independiente de informaci\'on
\cventry{\small{Marzo 2018 - Agosto 2018}}{Trabajos de Extensi\'on y Gu\'ia Tur\'istica en el Observatorio Astron\'omico Nacional}{Universidad de Chile}{}{}{}
\cventry{\small{Septiembre 2017 - Octubre 2017}}{Ayudante de laboratorio en Diplomado en Visualizaci\'on de Datos}{Pontificia Universidad Cat\'olica de Chile}{}{}{\break El/la ayudante se encarga de resolver dudas de forma presencial y guiar el trabajo complementario del curso.}
\cventry{\small{Marzo 2017 - Julio 2017}}{Profesora Auxiliar del curso de Mec\'anica}{Universidad de Chile}{}{}{\break El/ la auxiliar hace clases con ejercicios complementarios a la materia de c\'atedra y se encarga de tomar y revisar las evaluaciones a los alumnos.}
\cventry{\small{Enero 2017}}{Pr\'actica de Investigaci\'on}{Observatorio Astron\'omico Nacional}{Universidad de Chile}{}{\break An\'alisis datos de emisiones de mon\'oxido de carbono y obtenci\'on de las caracter\'isticas f\'isicas de una regi\'on de nubes moleculares en el Puente de Magallanes.} %REVISAR
\cventry{\small{Marzo 2015 - Diciembre 2016}}{Ayudante de laboratorio del curso de Introducci\'on a la Ingenier\'ia (I y II)}{Universidad de Chile}{}{}{\break El/la ayudante est\'a cargo de la evaluaci\'on y ense\~nanza de trabajo en taller de 2 a 3 grupos de 5 personas cada uno, por cada semestre. Se realiz\'o asistencia presencial a los grupos y evaluaci\'on en informes, trabajos pr\'acticos, resultados y trabajo en equipo.}
%\cventry{\small{Enero 2014}}{Asistente de Bodega}{Mediplex S.A.}{}{}{\break El/la asistente se encarga de la clasificaci\'on de insumos para su despacho y la emisi\'on de las facturas correspondientes.}

%------------------------------------------------


\section{Habilidades y calificaciones}
\cvitem{\textbf{\small{Idiomas}}}{Ingl\'es (TOEFL iBT: 113/120), Espa\~nol (Nativo)}
\cvitem{\textbf{\small{Programaci\'on}}}{{Python}, {Java}, {C}, {Matlab/Octave}, {R}, HTML}
\cvitem{\textbf{\small{Astronom\'ia}}}{CASA (Common Astronomy Software Applications), GILDAS, SAOImage DS9}
\cvitem{\textbf{\small{Otros}}}{GitHub, LaTeX, Google Drive}

%\section{Habilidades computacionales}

\cvitem{\textbf{\small{Lenguajes}}}{{Python}, {Java}, {C}, {Matlab/Octave}, {R}}
%\cvitem{\textbf{\small{SO}}}{Windows, Linux, OS X}
%\cvitem{\textbf{\small{Lenguajes de Script}}}{HTML}

%\cvitem{\textbf{\small{Programas de desarrollo}}}{Atom, Eclipse, Spyder, RStudio}
%\cvitem{\textbf{\small{Tesiting Tools}}} {JUnit}
%\cvitem{\textbf{\small{Version Controlling Tool}}}{GitHub URL - "https://github.com/tere-valdivia"}
%\cvitem{\textbf{\small{Otros}}}{CASA (Common Astronomy Software Applications), GILDAS, SAOImage DS9, Photoshop, LaTeX}


\section{Experiencia en difisi\'on cient\'ifica}
\cventry{\small{Julio 2018 - }}{Monitora “Cazadoras de Estrellas” }{DIRVEX, Universidad de Chile}{Talleres para mujeres en enseñanza media para que se familiaricen con el quehacer astronómico, sepan en qué consiste la carrera y puedan conocer mujeres astrónomas.}{}{}
\cventry{\small{27 de junio, 2019}}{Charla "Eclipses: Luces y Sombras en el Universo"}{Charla pública para enseñanza básica}{Municipalidad de El Bosque}{Santiago, Chile}{}
\cventry{\small{2 de marzo y 9 de abril, 2019}}{"Astroinform\'atica para principiantes"}{Charla para público general}{Centro de Extensi\'on de la Universidad de Talca (marzo) y DUOC UC Puente Alto (abril)}{Chile}{}
\cventry{\small{Junio a noviembre, 2018}}{Asesor científico “Festival de Teatro y Ciencias (FETYC)”}{Liceo Polivalente San Jos\'e de Maipo}{Par Explora Sur Oriente, CONICYT}{FETYC es una competencia de teatro y ciencias organizada por los Par Explora de la Región Metropolitana de Chile, donde los colegios participantes preparan una obra de teatro alrededor de un tema de interés científico. El asesor científico ayuda a los alumnos a preparar el contenido científico de la obra.}{}
\cventry{\small{Marzo a agosto, 2018}}{Gu\'ia Tur\'istico}{Observatorio Astron\'omico Nacional, Departamento de Astronom\'ia, Universidad de Chile}{Chile}{}{}

%----------------------------------------------------------------------------------------
%   AWARDS SECTION
%----------------------------------------------------------------------------------------
\section{Premios, becas y equipos de competencia}
\cvitem{2018}{Adelina Gutierrez Grant, Sociedad Chilena de Astronom\'ia}
\cvitem{2018-2019}{Beca de Mag\'ister Nacional, Comisi\'on Nacional de Investigaci\'on en Ciencia y Tecnolog\'ia (CONICYT), Chile}
\cvitem{2015-2016}{Alumno destacado, Licenciatura en Ciencias, Menci\'on Astronom\'ia}
\cvitem{2014}{Alumno destacado, Plan Com\'un}
\cvitem{2011-2012}{Preselecci\'on Nacional para las Olimpiadas de F\'isica}


\section{Actividades de Liderazgo}
\cventry{\small{Abril 2019 - Present}}{Representante Departamental Estudiantil}{Departamento de Astronom\'ia}{Universidad de Chile}{Santiago, Chile}{}
%\cventry{\small{Septiembre 2016 - Diciembre 2017}}{\textbf{Sustentabilidad}}{Delegada de Sustentabilidad de AFIAS, Centro de Estudiantes de Ingenier\'ia}{Universidad de Chile}{}{}
%\cventry{\small{Marzo 2017 - Julio 2018}}{\textbf{Juegos de Rol y de Mesa}}{Miembro de directiva del Club de Rol de InJenier\'ia}{Universidad de Chile}{}{}
\cventry{\small{2012}}{\textbf{Centro de Estudiantes}}{Miembro de la Comisi\'on de Servicio Social}{Santiago College}{}{}
\cventry{\small{2005-2012}}{\textbf{Scouts}}{Miembro del Grupo de Gu\'ias y Scouts del Santiago College}{Distrito Los Leones}{Santiago}{}

\section{Referencias}
%
\cventry{}{\textbf{M\'onica Rubio}}{Ph.D en Astrof\'isica}{Profesora Titular}{Departamento de Astronom\'ia, Universidad de Chile}{mrubio@das.uchile.cl}
%\cventry{}{\textbf{Patricio Cordero}}{Ph.D en F\'isica}{Profesor Titular}{Departamento de F\'isica, Universidad de Chile}{pcordero@ing.uchile.cl}
\cventry{}{\textbf{Mauricio Vargas}}{Data Analyst}{Datawheel LLC}{Cambridge, USA}{mvargas@dcc.uchile.cl}

%----------------------------------------------------------------------------------------
%   COMMUNICATION SKILLS SECTION
%----------------------------------------------------------------------------------------


    
\end{document}